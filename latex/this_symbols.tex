%%%%%%%%%%%%%%%%%%%%%%%%%%%%%%%%%%%%%%%%
%% Misc style stuff
%%%%%%%%%%%%%%%%%%%%%%%%%%%%%%%%%%%%%%%%
% make in NIPS format
\usepackage{nips10submit_e}
\nipsfinalcopy

\definecolor{mygray}{RGB}{153,153,153}
\definecolor{myred}{RGB}{221,77,57}
\definecolor{myblue}{RGB}{68,104,252}
\definecolor{mygreen}{RGB}{86,149,22}


% \raisebox{distance}[extend-above][extend-below]{text}
\newcommand*{\rsquare}{ \textcolor{myred}{\rule{0.14in}{0.14in}} }
\newcommand*{\graysquare}{ \raisebox{-2pt}{\textcolor{mygray}{\rule{0.14in}{0.14in}}} }
\newcommand*{\bsquare}{ \textcolor{myblue}{\rule{0.14in}{0.14in}} }
\newcommand*{\gsquare}{ \raisebox{-2pt}{\textcolor{mygreen}{\rule{0.14in}{0.14in}}} }
%\newcommand*{\ysquare}{ \textcolor{yellow}{\rule{0.1in}{0.1in}} }

\newcommand*{\balpha}{\boldsymbol{\alpha}}
\newcommand*{\bbeta}{\boldsymbol{\beta}}

\newcommand*{\wsquare}{ \rule{0.1in}{0.1in} }

%\newcommand*{\rsquare}{ \textcolor{red}{$\CIRCLE$} }
%\newcommand*{\bsquare}{ \textcolor{blue}{$\CIRCLE$} }
%\newcommand*{\ysquare}{ \textcolor{yellow}{$\CIRCLE$} }

\newcommand*{\rcircle}{ \textcolor{red}{$\CIRCLE$}  \textcolor{black}{$\Circle$}  }

%%%%%%%%%%%%%%%%%%%%%%%%%%%%%%%%%%%%%%%%
%% Commands only for this paper
%%%%%%%%%%%%%%%%%%%%%%%%%%%%%%%%%%%%%%%%
%\newcommand*{\X}{\ensuremath{\mathcal{X}}\xspace}
%\newcommand*{\x}{\ensuremath{\mathcal{x}}}

\newcommand*{\X}{\ensuremath{X_{1 \ldots n}\xspace}}
%\newcommand*{\x}{\ensuremath{x}\xspace}

%\renewcommand*{\P}{\ensuremath{\mathcal{P}} }
\renewcommand*{\P}{\ensuremath{\mathbf{P}}\xspace}
\newcommand*{\A}{\ensuremath{\mathbf{\mathcal{A}}}}
\newcommand*{\R}{\ensuremath{\textbf{R}} }

%\newcommand*{\Psize}{\ensuremath{\mid\P\mid} }
\newcommand*{\Psize}{\ensuremath{n}\xspace}
%\newcommand*{\Psize}{\ensuremath{\eta} }

\newcommand*{\perm}{\operatorname{\textsc{perm}}}
\newcommand*{\holism}{\textnormal{\Pisces}\xspace}
%\newcommand*{\holism}{\circledS\xspace}
%\newcommand*{\holism}{ \ensuremath{\integ_{\!\mathcal{H}} } }
%\newcommand*{\integ}{\!\!\textnormal{ {\Gemini}}}


\newcommand*{\integ}{ \ensuremath{\Upsilon^{\textnormal{prf}}}\xspace}
%\newcommand*{\integ}{ \ensuremath{\mathcal{CC}} }
%\newcommand*{\integ}{ \ensuremath{\Gamma} }
\newcommand*{\integir}{ \ensuremath{\Upsilon}\xspace}

%\newcommand*{\integ}{ \raisebox{7pt}{\rotatebox{180}{\ensuremath{\Upsilon}}}}

%\newcommand*{\integ}{ \raisebox{7pt}{\rotatebox{180}{ \Aries }} \! \! }
%\newcommand*{\integ}{ \textnormal{\Gemini } }

\newcommand*{\TSE}{\textnormal{\textsc{TSE-comp}}}

% For when discussing the operators within the text.
%\newcommand*{\XOR}{\ensuremath{\operatorname{XOR}}}
%\newcommand*{\AND}{\ensuremath{\operatorname{AND}}}
%\newcommand*{\GET}{\ensuremath{\operatorname{GET}}}


%\newcommand*{\beiP}{\ensuremath{\langle {\ei}(\P)\rangle}}
\newcommand*{\holismP}{ \ensuremath{ \holism \left( \P_0 : \X_1 \right) }\xspace}
%\newcommand*{\integP}{ \ensuremath{ \integ_\P \left( X_1 \right) } }

\newcommand*{\integP}{ \ensuremath{ \mathcal{CC}\left( \P_0 : \X_1 \right) }\xspace}

%\newcommand*{\eiP}{\ensuremath{ {\ei}(\P) }}

% define a format for printing BINARY numbers
\newcommand*{\bin}[1]{\texttt{#1}}

\newcommand*{\PI}{\ensuremath{\operatorname{I}_{\partial} }\!\xspace}
%\newcommand*{\PI}{ \scalebox{1.15}{\ensuremath{ \operatorname{I}_{\partial} } } }

%\newcommand*{\ei}{\ensuremath{{\tt ei}}\xspace}

\newcommand*{\beiP}{ \ensuremath{ \ei \! \left( \X_0 \rightarrow \X_1 / \P \right) }\xspace}
\newcommand*{\eiP}{ \ensuremath{ \ei \! \left( \X_1 / \P \right) }\xspace}
%\newcommand*{\ei}[1]{ \ei \! \left( #1 \right) }

\renewcommand*{\Imin}[2]{\operatorname{I}_{\min} \! \left( #1 \!:\! #2 \right)}

\newcommand*{\WMS}{\operatorname{WMS}}
\newcommand*{\minWMS}{\operatorname{WMMS}}
\renewcommand*{\S}{\ensuremath{\mathcal{S}}\xspace}
\renewcommand*{\R}{\ensuremath{\mathcal{R}}\xspace}

\newcommand*{\U}{\ensuremath{\mathcal{U}}\xspace}

\newcommand*{\setX}{\ensuremath{\mathbf{X}}\xspace}
\newcommand*{\setx}{\ensuremath{\mathbf{x}}\xspace}

% For the section on \Delta I
%\newcommand*{\ind}{\textnormal{ind}}

%\renewcommand*{\Sop}[2]{\operatornamewithlimits{\mathcal{S}}\left( #1 : #2 \right)}
\newcommand*{\ei}{\ensuremath{{\tt ei}}\xspace}
\renewcommand*{\=}{ \ensuremath{ \! = \!}}

\renewcommand*{\flat}[1]{\ensuremath{\overline{#1}}}


\newcommand*{\unq}[1]{\ensuremath{\opname{unq} \! \left( #1 \right)\xspace}}
%\newcommand*{\Icap}{\ensuremath{\opname{I}_{\cap} \xspace}}
%\newcommand*{\Isyn}{\ensuremath{\opname{I}_{\mathcal{S}} \xspace}}
%\newcommand*{\ind}{\ensuremath{\textnormal{ind}}}
%\newcommand*{\opI}{\ensuremath{ \opname{I}\xspace}}

%\textstyle{\sum}
\newcommand*{\Probstar}[1]{ \ensuremath{ \textstyle{\Pr^*} \!\! \left( #1 \right)}}

\newcommand*{\infostar}[2]{ \ensuremath{ \textstyle{\opI^*} \!\! \left( #1 \! : \! #2 \right)}}


%\newcommand*{\Icup}{\ensuremath{\opname{I}_{\cup} \xspace}}

\newcommand*{\IHo}{ \ensuremath{\opname{I}_{\textsc{Ho}}}\xspace}
\newcommand*{\IHoe}[2]{ \IHo\left(#1 : #2 \right)}

\newcommand*{\Ismp}{\ensuremath{\opname{I}_{\opname{smp} \xspace}}}


\newcommand*{\Ialpha}{\ensuremath{\opname{I}_{\alpha} \xspace}}
\newcommand*{\Ibeta}{\ensuremath{\opname{I}_{\beta} \xspace}}

\newcommand*{\IalphaZ}{\ensuremath{\opname{I}_{\alpha}^0 \xspace}}
\newcommand*{\IbetaZ}{\ensuremath{\opname{I}_{\beta}^0 \xspace}}


\newcommand*{\Ialphae}[2]{\ensuremath{\opname{I}_{\alpha}\left( #1 : #2 \right)}}
\newcommand*{\Ibetae}[2]{\ensuremath{\opname{I}_{\beta}\left( #1 : #2 \right) }}

\newcommand*{\IalphaZe}[2]{\ensuremath{\opname{I}_{\alpha}^0\left( #1 : #2 \right)}}
\newcommand*{\IbetaZe}[2]{\ensuremath{\opname{I}_{\beta}^0\left( #1 : #2 \right) }}

\newcommand*{\chkm}{\checkmark}

